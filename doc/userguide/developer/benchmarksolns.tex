
\chapter{\label{cha:Analytical-Solns}Analytical Benchmark Solutions}


\section{\label{sec:PoroelasticProblems}Poroelastic Problems}

\subsection{\label{sub:Terzaghi-Consolidation}Terzaghi's Consolidation Problem}

A one dimensional poroelastic analytical problem may be arrived at
by reviewing the consolidation problem of Terzaghi\cite{Terzaghi:1923} in light
of Biot theory. An initially undisturbed soil layer of thickness L, resting upon
a rigid, impermeable base is considered, and a constant load is applied to the top
layer. All sides are held to be no flux boundaries, with the exception of the top,
which is taken to represent drained conditions. At $t = 0^{+}$, the sample compacts,
and the sudden applied load causes the pore pressure to spike to the undrained 
value, also known as the Skempton effect. The physical conditions may be expressed 
mathematically as 
\begin{align}
 \sigma_{zz} = -P_{0}H(t), \quad p = 0; \qquad \text{at} \quad z &=0; \qquad t \geq 0
 u_{z} = 0, \quad \frac{\partial p}{\partial z} = 0; \qquad \text{at} \quad z &= L; \quad t \geq 0
\end{align}

The pressure distribution over time for drained conditions may be represented with a closed form, analytical
solution:

\begin{equation}
 p(z,t) = P_{0} \sum_{n = 1,3,\ldots}^{\infty}
\end{equation}



\subsection{Mandel's Problem}



The analytical solution for pore pressure, generalized for the case of compressible
constituents may be expressed as\cite{Cheng:Detourany:1988}:

\begin{equation}
 p = \frac{1}{A_{2} - A_{1}}   \sum_{i=1}^{\infty} D_{i} \left[ \cos \left( \frac{\alpha_{i}x}{a} \right) - \cos \alpha_{i} \right] e^{\frac{-\alpha_{i}^{2} c_{x} t}{a^{2}}} 
\end{equation}



