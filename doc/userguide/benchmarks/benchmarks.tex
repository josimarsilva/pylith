\chapter{Benchmarks}
\label{sec:benchmarks}

\warning{None of the benchmark input files in the PyLith benchmarks
  repository on GitHub have been updated for v3.0.}

\section{Overview}

The Crustal Deformation Modeling and Earthquake Source Physics Focus
Groups within the Southern California Earthquake Center and the Short-Term
Tectonics Working Group within CIG have developed a suite of benchmarks
to test the accuracy and performance of 3D numerical codes for quasi-static
crustal deformation and earthquake rupture dynamics. The benchmark
definitions for the quasi-static crustal deformation benchmarks are
posted on the CIG website at Short-Term Tectonics Benchmarks \url{geodynamics.org/cig/workinggroups/short/workarea/benchmarks/}
and the definitions for the earthquake rupture benchmarks are posted
on the SCEC website \url{scecdata.usc.edu/cvws/cgi-bin/cvws.cgi}.
This suite of benchmarks permits evaluating the relative performance
of different types of basis functions, quadrature schemes, and discretizations
for geophysical applications. The files needed to run the 3D benchmarks
are in the CIG GitHub Repository \url{https://github.com/geodynamics/pylith_benchmarks}.
In addition to evaluating the efficiency and accuracy of numerical
codes, the benchmarks also make good test problems, where users can
perform simulations based on actual geophysical problems. The benchmarks
are performed at various resolutions and using different element types.
By comparing the runtime and accuracy for different resolutions and
element types, users can evaluate which combination will be best for
their problems of interest.

\input{benchmarks/quasistatic_strikeslip.tex}
\input{benchmarks/savageprescott.tex}


\section{SCEC Dynamic Rupture Benchmarks}
\label{sec:scec:dynamic:rupture:benchmarks}

The SCEC website \url{scecdata.usc.edu/cvws/cgi-bin/cvws.cgi} includes
a graphical user interface for examining the benchmark results. Benchmark
results for PyLith are available for TPV205-2D (horizontal slice through
a vertical strike-slip fault), TPV205 (vertical strike-slip fault
with high and low stress asperities), TPV210-2D (vertical slice through
a 60-degree dipping normal fault), TPV210 (60-degree dipping normal
fault), TPV11, TPV12, TPV13, TPV14-2D and TPV15-2D (horizontal slice
through a verticel strike-slip fault with a branch), TPV14, TPV15,
TPV 24, TPV25 (vertical strike-slip fault with a branch), TPV 16 and
17 (vertical strike-slip fault with spatially heterogeneous initial
tractions), TPV 22 and 23 (vertical strike-slip fault with a stepover),
TPV102 (vertical strike-slip fault with rate-state friction).

The benchmark results indicate that triangular and tetrahedral cells
generate less numerical noise than quadrilateral or hexahedral cells.
The input files in the repository are updated for PyLith v2.0.0, so
you will need to modify them if you use another version of PyLith.
