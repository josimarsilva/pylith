\section{Elasticity With Infinitesimal Strain and No Faults}

We begin with the elasticity equation including the inertial term,
\begin{gather}
  \label{eqn:elasticity:strong:form}
  \rho \frac{\partial^2\vec{u}}{\partial t^2} - \vec{f}(\vec{x},t) - \tensor{\nabla} \cdot 
\tensor{\sigma}
(\vec{u}) = \vec{0} \text{ in }\Omega, \\
%
  \label{eqn:bc:Neumann}
  \tensor{\sigma} \cdot \vec{n} = \vec{\tau}(\vec{x},t) \text{ on }\Gamma_\tau, \\
%
  \label{eqn:bc:Dirichlet}
  \vec{u} = \vec{u}_0(\vec{x},t) \text{ on }\Gamma_u,
\end{gather}
where $\vec{u}$ is the displacement vector, $\rho$ is the mass
density, $\vec{f}$ is the body force vector, $\tensor{\sigma}$ is the
Cauchy stress tensor, $\vec{x}$ is the spatial coordinate, and $t$ is
time. We specify tractions $\vec{\tau}$ on boundary $\Gamma_\tau$, and
displacements $\vec{u}_0$ on boundary $\Gamma_u$. Because both $\vec{\tau}$
and $\vec{u}$ are vector quantities, there can be some spatial overlap
of boundaries $\Gamma_\tau$ and $\Gamma_u$; however, a degree of freedom at
any location cannot be associated with both prescribed displacements
(Dirichlet) and traction (Neumann) boundary conditions simultaneously.

\subsection{Notation}
\begin{itemize}
\item Unknowns
  \begin{description}
  \item[$\vec{u}$] Displacement field
  \item[$\vec{v}$] Velocity field (if including inertial term)
  \end{description}
\item Derived quantities
  \begin{description}
    \item[$\tensor{\sigma}$] Cauchy stress tensor
    \item[$\tensor{\epsilon}$] Cauchy strain tensor
  \end{description}
\item Constitutive parameters
  \begin{description}
  \item[$\mu$] Shear modulus
  \item[$K$] Bulk modulus
  \item[$\rho$] Density
  \end{description}
\item Source terms
  \begin{description}
    \item[$\vec{f}$] Body force per unit volume, for example $\rho \vec{g}$
  \end{description}
\end{itemize}

\subsection{Neglecting Inertia}

If we neglect the inertial term, then time dependence only arises
from history-dependent constitutive equations and boundary
conditions. Considering the displacement $\vec{u}$ as the unknown, we
have
\begin{align}
  \vec{s}^T &= (\vec{u})^T, \\
%
  \vec{0} &= \vec{f}(\vec{x},t) + \tensor{\nabla} \cdot \tensor{\sigma}(\vec{u}) \text{ in }
\Omega, \\
% Neumann
  \tensor{\sigma} \cdot \vec{n} &= \vec{\tau}(\vec{x},t) \text{ on }\Gamma_\tau, \\
% Dirichlet
  \vec{u} &= \vec{u}_0(\vec{x},t) \text{ on }\Gamma_u.
\end{align}
We create the weak form by taking the dot product with the trial
function $\trialvec[u]$ and integrating over the domain:
\begin{equation}
  0 = \int_\Omega \trialvec[u] \cdot \left( \vec{f}(t) + \tensor{\nabla} \cdot \tensor{\sigma}
(\vec{u})  \right) 
\, d\Omega.
\end{equation}
Using the divergence theorem and incorporating the Neumann bounday
condition yields
\begin{equation}
  0 = \int_\Omega \trialvec[u] \cdot \vec{f}(t) + \nabla \trialvec[u] : -\tensor{\sigma}
(\vec{u}) \, d\Omega + 
\int_{\Gamma_\tau} \trialvec[u] \cdot \vec{\tau}(\vec{x},t) \, d\Gamma.
\end{equation}

Identifying $F(t,s,\dot{s})$ and $G(t,s)$, we have
\begin{alignat}{2}
  F^u(t,s,\dot{s}) &= \vec{0},
  & \qquad
  G^u(t,s) &= \int_\Omega \trialvec[u] \cdot \eqnannotate{\vec{f}(\vec{x},t)}{g_0^u} + \nabla 
\trialvec[u] : 
\eqnannotate{-\tensor{\sigma}(\vec{u})}{g_1^u} \, d\Omega + \int_{\Gamma_\tau} \trialvec[u] 
\cdot 
\eqnannotate{\vec{\tau}(\vec{x},t)}{g_0^u} \, d\Gamma.
\end{alignat}


\subsubsection{Jacobians}

With the solution composed of the displacement field and no LHS function, we only have 
Jacobians for the RHS,
\begin{align}
  J_G^{uu} &= \frac{\partial G^u}{\partial u} = \int_\Omega \nabla \trialvec[u] : 
\frac{\partial}{\partial u}(-
\tensor{\sigma}) \, d\Omega 
  = \int_\Omega \nabla \trialvec[u] : -\tensor{C} : \frac{1}{2}(\nabla + \nabla^T)\basisvec[u] 
\, d\Omega 
  = \int_\Omega \trialscalar[v]_{i,k} \, \eqnannotate{\left( -C_{ikjl} \right)}{J_{g3}^{uu}}  
\, 
\basisscalar[u]_{j,l}\, d\Omega
\end{align}

\subsection{Including Inertia}

For convenience we cast the elasticity equation in the form of a first order
equation by considering both the displacement $\vec{u}$ and velocity $\vec{v}$
as unknowns,
\begin{align}
  \vec{s}^T &= (\vec{u} \quad \vec{v})^T, \\
%
  \label{eqn:velocity:strong:form}
  \frac{\partial\vec{u}}{\partial t} &= \vec{v}, \\
%
  \label{eqn:elasticity:order1:strong:form}
  \rho \frac{\partial\vec{v}}{\partial t} &= \vec{f}(\vec{x},t) + \tensor{\nabla} \cdot 
\tensor{\sigma}(\vec{u}) 
\text{ in }\Omega, \\
% Neumann
  \tensor{\sigma} \cdot \vec{n} &= \vec{\tau}(\vec{x},t) \text{ on }\Gamma_\tau, \\
% Dirichlet
  \vec{u} &= \vec{u}_0(\vec{x},t) \text{ on }\Gamma_u.
\end{align}

For trial functions $\trialvec[u]$ and $\trialvec[v]$ we write the weak form as
\begin{align}
  \int_\Omega \trialvec[u] \cdot \left( \frac{\partial \vec{u}}{\partial t} \right) \, d\Omega 
&= 
  \int_\Omega \trialvec[u] \cdot \vec{v} \, d\Omega, \\
%
  \int_\Omega \trialvec[v] \cdot \left( \rho \frac{\partial \vec{v}}{\partial t} \right) \, 
d\Omega &= 
  \int_\Omega \trialvec[v] \cdot \left( \vec{f}(\vec{x},t) + \tensor{\nabla} \cdot 
\tensor{\sigma}(\vec{u})  
\right) \, d\Omega.
%
\end{align}
Using the divergence theorem and incorporating the Neumann boundary
conditions, we can rewrite the second equation as
\begin{equation}
  \label{eqn:elasticity:displacement}
  \int_\Omega \trialvec[v] \cdot \left( \rho \frac{\partial \vec{v}}{\partial t} \right) \, 
d\Omega =
  \int_\Omega \trialvec[v] \cdot \vec{f}(\vec{x},t) + \nabla \trialvec[v] : -\tensor{\sigma}
(\vec{u}) \, d\Omega + 
\int_{\Gamma_\tau} \trialvec[v] \cdot \vec{\tau}(\vec{x},t) \, d\Gamma.
\end{equation}

% ----------------------------------------------------------------------
\subsubsection{Implicit Time Stepping}
In practice we do not use implicit time stepping when we include
inertia. We provide this section to illustrate the derivation of the
point-wise functions for the residual and Jacobian. The resulting
system of equations to solve is
\begin{align}
  \label{eqn:elasticity:velocity:implicit}
  \int_\Omega \trialvec[u] \cdot \left( \frac{\partial \vec{u}}{\partial t} \right) \, d\Omega 
&= 
  \int_\Omega \trialvec[u] \cdot \vec{v} \, d\Omega, \\
%
  \label{eqn:elasticity:displacement:implicit}
  \int_\Omega \trialvec[v] \cdot \left( \rho \frac{\partial \vec{v}}{\partial t} \right) \, 
d\Omega &=
  \int_\Omega \trialvec[v] \cdot \vec{f}(\vec{x},t) + \nabla \trialvec[v] : -\tensor{\sigma}
(\vec{u}) \, d\Omega + 
\int_{\Gamma_\tau} \trialvec[u] \cdot \vec{\tau}(\vec{x},t) \, d\Gamma.
\end{align}
Identifying $F(t,s,\dot{s})$ and $G(t,s)$, we have
\begin{alignat}{2}
  F^u(t,s,\dot{s}) &= \int_\Omega \trialvec[u] \cdot \eqnannotate{\left( \frac{\partial 
\vec{u}}{\partial t} 
\right)}{f_0^u} \, d\Omega,
  & \qquad
  G^u(t,s) &= \int_\Omega \trialvec[u] \cdot \eqnannotate{\vec{v}}{g_0^u} \, d\Omega, \\
  %  
  F^v(t,s,\dot{s}) &= \int_\Omega \trialvec[v] \cdot \eqnannotate{\left( \rho \frac{\partial 
\vec{v}}{\partial t} 
\right)}{f_0^v} \, d\Omega,
  & \qquad
  G^v(t,s) &= \int_\Omega \trialvec[v] \cdot \eqnannotate{\vec{f}(\vec{x},t)}{g_0^v} + \nabla 
\trialvec[v] : 
\eqnannotate{-\tensor{\sigma}(\vec{u})}{g_1^v} \, d\Omega + \int_{\Gamma_\tau} \trialvec[u] 
\cdot 
\eqnannotate{\vec{\tau}(\vec{x},t)}{g_0^v} \, d\Gamma.
\end{alignat}


\subsubsection{Jacobians}

With two fields we have four Jacobians for each side of the equation associated with the 
coupling of the two 
fields,
\begin{align}
  J_F^{uu} &= \frac{\partial F^u}{\partial u} + s_\mathit{tshift} \frac{\partial F^u}{\partial 
\dot{u}} = \int_\Omega 
\trialvec[u] \cdot s_\mathit{tshift}\,\basisvec[u] \, d\Omega = \int_\Omega \trialscalar[u]_i 
\, 
\eqnannotate{s_\mathit{tshift} \delta_{ij}}{J_{f0}^{uu}} \, \basisscalar[u]_j \, d\Omega \\
  J_F^{uv} &= \frac{\partial F^u}{\partial v} + s_\mathit{tshift} \frac{\partial F^u}{\partial 
\dot{v}} = \tensor{0} \\
  J_F^{vu} &= \frac{\partial F^v}{\partial u} + s_\mathit{tshift} \frac{\partial F^v}{\partial 
\dot{u}} = \tensor{0} \\
  J_F^{vv} &= \frac{\partial F^v}{\partial v} + s_\mathit{tshift} \frac{\partial F^v}{\partial 
\dot{v}} = \int_\Omega 
\trialvec[v] \cdot s_\mathit{tshift}\,\rho\,\basisvec[v] \, d\Omega = \int_\Omega 
\trialscalar[v]_i \, 
\eqnannotate{s_\mathit{tshift} \, \rho \, \delta_{ij}}{J_{f0}^{vv}} \, \basisscalar[v]_j \, 
d\Omega \\
  J_G^{uu} &= \frac{\partial G^u}{\partial u} = \tensor{0} \\
  J_G^{uv} &= \frac{\partial G^u}{\partial v} = \int_\Omega \trialvec[u] \cdot \basisvec[v] \, 
d\Omega = 
\int_\Omega \trialscalar[u]_i \, \eqnannotate{\delta_{ij}}{J_{g0}^{uv}} \, \basisscalar[v]_j 
\, d\Omega \\
  J_G^{vu} &= \frac{\partial G^v}{\partial u} = \int_\Omega \nabla \trialvec[v] : 
\frac{\partial}{\partial u}(-
\tensor{\sigma}) \, d\Omega 
  = \int_\Omega \nabla \trialvec[v] : -\tensor{C} : \frac{1}{2}(\nabla + \nabla^T)\basisvec[u] 
\, d\Omega 
  = \int_\Omega \trialscalar[v]_{i,k} \, \eqnannotate{\left( -C_{ikjl} \right)}{J_{g3}^{vu}}  
\, 
\basisscalar[u]_{j,l}\, d\Omega \\
  J_G^{vv} &= \frac{\partial G^v}{\partial v} = \tensor{0}
\end{align}

% ----------------------------------------------------------------------
\subsection{Explicit Time Stepping}
Recall that explicit time stepping requires $F(t,s,\dot{s})=\dot{s}$. We write $F^*(t,s,
\dot{s}) = \dot{s}$ and
$G^*(t,s) = J_F^{-1}G(t,s)$ and we do not provide functions for $f_0$ and $f_1$. Thus, our 
system of equations to 
solve is
\begin{align}
  \label{eqn:elasticity:velocity:explicit}
  \int_\Omega \trialvec[u] \cdot \frac{\partial \vec{u}}{\partial t} \, d\Omega &= 
  \int_\Omega \trialvec[u] \cdot \vec{v} \, d\Omega, \\
%
  \label{eqn:elasticity:displacement:explicit}
  \int_\Omega \trialvec[v] \cdot \frac{\partial \vec{v}}{\partial t} \, d\Omega &=
  \frac{1}{\int_\Omega \trialvec[v] \cdot \rho\,\basisvec[v] \, d\Omega} \left( \int_\Omega 
\trialvec[v] \cdot 
\vec{f}(\vec{x},t) + \nabla \trialvec[u] : -\tensor{\sigma}(\vec{u}) \, d\Omega + 
\int_{\Gamma_\tau} \trialvec[u] 
\cdot \vec{\tau}(\vec{x},t) \, d\Gamma \right).
\end{align}
Identifying $F(t,s,\dot{s})$ and $G(t,s)$, we have
\begin{align}
  F^u(t,s,\dot{s}) &= \int_\Omega \trialvec[u] \cdot \frac{\partial \vec{u}}{\partial t} \, 
d\Omega, \\
%
  G^u(t,s) &= \int_\Omega \trialvec[u] \cdot \eqnannotate{\vec{v}}{g_0^u} \, d\Omega, \\
  %  
  F^v(t,s,\dot{s}) &= \int_\Omega \trialvec[v] \cdot \frac{\partial \vec{v}}{\partial t}  \, 
d\Omega, \\
%
  G^v(t,s) &= \frac{1}{\int_\Omega \trialvec[v] \cdot {\eqnannotate{\rho}{J_{f0}^{vv}}}
\basisvec[v] \, d\Omega} 
\left( \int_\Omega \trialvec[v] \cdot \eqnannotate{\vec{f}(t)}{g_0^v} + \nabla \trialvec[v] : 
\eqnannotate{-
\tensor{\sigma}(\vec{u})}{g_1^v} \, d\Omega + \int_{\Gamma_\tau} \trialvec[v] \cdot 
\eqnannotate{\vec{\tau}
(\vec{x},t)}{g_0^v} \, d\Gamma \right).
\end{align}
where $J_{f0}^{uu} = \tensor{I}$, and we refer to $J_F$ as the LHS
(or I) Jacobian for explicit time stepping.

% ----------------------------------------------------------------------
\subsection{Elasticity Constitutive Models}

The Jacobian for the elasticity equation is
\begin{equation}
J_{G}^{vu} = \frac{\partial G^{v_i}}{\partial u_j}.
\end{equation}
In computing the derivative, we consider the linearized form:
\begin{align}
  \sigma_{ik} &= C_{ikjl} \epsilon_{jl} \\
  \sigma_{ik} &= C_{ikjl} \frac{1}{2} ( u_{j,l} + u_{l,j} ) \\
  \sigma_{ik} &= \frac{1}{2} ( C_{ikjl} + C_{iklj} ) u_{j,l} \\
  \sigma_{ik} &= C_{ikjl} u_{j,l} \\
\end{align}
In computing the Jacobian, we take the derivative of the stress tensor with respect to the 
displacement field,
\begin{equation}
  \frac{\partial}{\partial u_j} \sigma_{ik} = C_{ikjl} \basisscalar[u]_{j,l},
\end{equation}
so we have
\begin{equation}
\boxed{
  J_{g3}^{vu}(i,j,k,l) = -C_{ikjl}
}
\end{equation}
For many elasticity constitutive models we prefer to separate the
stress into the mean stress and deviatoric stress:
\begin{gather}
  \tensor{\sigma} = \sigma^\mathit{mean} \tensor{I} + \tensor{\sigma}^\mathit{dev} \text{, 
where}\\
  \sigma^\mathit{mean} = \frac{1}{3} \Tr(\tensor{\sigma}) = \frac{1}{3} (\sigma_{11} + 
\sigma_{22} + \sigma_{33}).
\end{gather}
Sometimes it is convenient to use pressure (positive pressure corresponds to compression) 
instead of the mean 
stress:
\begin{gather}
  \tensor{\sigma} = -p \tensor{I} + \tensor{\sigma}^\mathit{dev} \text{, where}\\
  p = -\frac{1}{3} \Tr(\tensor{\sigma}).
\end{gather}

The Jacobian with respect to the deviatoric stress is
\begin{align}
  \frac{\partial \sigma^\mathit{dev}_{ik}}{\partial u_j}  &= \frac{\partial}{\partial u_j} 
\left(\sigma_{ik} - 
\frac{1}{3} \sigma_{mm} \delta_{ik} \right) \\
  \frac{\partial \sigma^\mathit{dev}_{ik}}{\partial u_j}  &= C_{ikjl} \basisscalar[u]_{j,l} - 
\frac{1}{3} C_{mmjl} 
\delta_{ik} \basisscalar[u]_{j,l}.
\end{align}
We call these modified elastic constants $C^\mathit{dev}_{ikjl}$, so that we have
\begin{equation}
\boxed{
  C^\mathit{dev}_{ikjl} = C_{ikjl} - \frac{1}{3} C_{mmjl} \delta_{ik}.
}
\end{equation}.

% ----------------------------------------------------------------------
\subsubsection{Isotropic Linear Elasticity}

We implement isotropic linear elasticity both with and without a
reference stress-strain state. With a linear elastic material it is
often convenient to compute the deformation relative to an unknown
initial stress-strain state. Furthermore, when we use an initial
undeformed configuration with zero stress and strain, the reference
stress and strain are zero, so this presents a simplifcation of the
more general case of the stress-strain state relative to the reference
stress-strain state.

Without a reference stress-strain state, we have
\begin{equation}
  \sigma_{ij} = \lambda \epsilon_{kk} \delta_{ij} + 2\mu\epsilon_{ij},
\end{equation}
and with a reference stress-strain state, we have
\begin{equation}
  \sigma_{ij} = \sigma_{ij}^\mathit{ref} + \lambda \left(\epsilon_{kk} - \epsilon_{kk}
^\mathit{ref}\right)
\delta_{ij} + 2\mu\left(\epsilon_{ij}-\epsilon_{ij}^\mathit{ref}\right).
\end{equation}
The mean stress is
\begin{align}
  \sigma^\mathit{mean} &= \frac{1}{3} \sigma_{kk}, \\
  \sigma^\mathit{mean} &= \frac{1}{3} \sigma_{kk}^\mathit{ref} + \left(\lambda+\frac{2}
{3}\mu\right)
\left(\epsilon_{kk}-\epsilon_{kk}^\mathit{ref}\right),
\end{align}
\begin{equation}
  \boxed{
  \sigma^\mathit{mean} = \frac{1}{3} \sigma_{kk}^\mathit{ref} + K \left(\epsilon_{kk}-
\epsilon_{kk}^\mathit{ref}
\right),
}%boxed
\end{equation}
where $K=\lambda+2\mu/3$ is the bulk modulus. 
If the reference stress and reference strain are both zero, then this reduces to
\begin{equation}
  \boxed{
  \sigma^\mathit{mean} = K \epsilon_{kk}.
}%boxed
\end{equation}
The deviatoric stress is
\begin{align}
  \sigma_{ij}^\mathit{dev} &= \sigma_{ij} - \sigma^\mathit{mean}\delta_{ij}, \\
  \sigma_{ij}^\mathit{dev} &= \sigma_{ij}^\mathit{ref} + \lambda\left(\epsilon_{kk}-
\epsilon_{kk}^\mathit{ref}
\right)\delta_{ij} + 2\mu\left(\epsilon_{ij}-\epsilon_{ij}^\mathit{ref}\right) - 
\left(\frac{1}{3}\sigma_{kk}
^\mathit{ref} + \left(\lambda+\frac{2}{3}\mu\right)\left(\epsilon_{kk}-\epsilon_{kk}
^\mathit{ref}\right)\right)
\delta_{ij}, \\
  \sigma_{ij}^\mathit{dev} &= \sigma_{ij}^\mathit{ref} -\frac{1}{3}\sigma_{kk}^\mathit{ref}
\delta_{ij} + 
2\mu\left(\epsilon_{ij}-\epsilon_{ij}^\mathit{ref}\right) - \frac{2}{3}\mu\left(\epsilon_{kk}-
\epsilon_{kk}
^\mathit{ref}\right)\delta_{ij},
\end{align}
\begin{equation}
  \boxed{
  \sigma_{ij}^\mathit{dev} = \left\{ \begin{array}{lcr}
      \sigma_{ii}^\mathit{ref} -\frac{1}{3}\sigma_{kk}^\mathit{ref} + 2\mu\left(\epsilon_{ii}-
\epsilon_{ii}
^\mathit{ref}\right) - \frac{2}{3}\mu\left(\epsilon_{kk}-\epsilon_{kk}^\mathit{ref}\right) & 
\text{if} & i = j, \\
      \sigma_{ij}^\mathit{ref} + 2\mu\left(\epsilon_{ij}-\epsilon_{ij}^\mathit{ref}\right) & 
\text{if} & i \neq j.
    \end{array} \right.
}%boxed
\end{equation}
If the reference stress and reference strain are both zero, then this reduces to
\begin{equation}
  \boxed{
  \sigma_{ij}^\mathit{dev} = \left\{ \begin{array}{lcr}
      2\mu\epsilon_{ii} - \frac{2}{3}\mu\epsilon_{kk} & \text{if} & i = j, \\
      2\mu\epsilon_{ij} & \text{if} & i \neq j.
    \end{array} \right.
  }%boxed
\end{equation}

For isotropic linear elasticity
\begin{align}
  C_{1112} &= C_{1113} = C_{1113} = C_{1121} = C_{1123} = C_{1131} = C_{1132} = 0\\
  C_{1211} &= C_{1213} = C_{1222} = C_{1223} = C_{1231} = C_{1232} = C_{1233} = 0,
\end{align}
and
\begin{align}
  C_{1111} = C_{2222} = C_{3333} &= \lambda + 2 \mu, \\
  C_{1122} = C_{1133} = C_{2233} &= \lambda, \\
  C_{1212} = C_{2323} = C_{1313} &= \mu.
\end{align}
The deviatoric elastic constants are:
\begin{align}
  C^\mathit{dev}_{1111} = C^\mathit{dev}_{2222} = C^\mathit{dev}_{3333} &= \frac{4}{3}\mu, \\
  C^\mathit{dev}_{1122} = C^\mathit{dev}_{1133} = C^\mathit{dev}_{2233} &= -\frac{2}{3}\mu, \\
  C^\mathit{dev}_{1212} = C^\mathit{dev}_{2323} = C^\mathit{dev}_{1313} &= \mu.
\end{align}

\subsubsection{Isotropic Generalized Maxwell Viscoelasticity}

We use the same general formulation for both the simple Maxwell
viscoelastic model and the generalized Maxwell model (several Maxwell
models in parallel). We implement the Maxwell models both with and
without a reference stress-strain state. Note that it is also possible
to specify an initial state variable value (viscous strain). Viscous
flow is completely deviatoric, so we split the stress into volumetric
and deviatoric parts, as described above.  The volumetric part is
identical to that of an isotropic elastic material. The deviatoric
part is given by:
\begin{equation}
  \sigma^\mathit{dev}_{ij}\left(t\right)=2\mu_{tot}\left(\mu_{0}\epsilon^\mathit{dev}_{ij}
    \left(t\right)+\sum_{m=1}^{N}\mu_{m}h^{m}_{ij}\left(t\right)-\epsilon^\mathit{refdev}_{ij}
    \right)+\sigma^\mathit{refdev}_{ij},
\end{equation}
where $\mu_{tot}$ is the total shear modulus of the model, $\mu_{0}$
is the fraction of the shear modulus accommodated by the elastic
spring in parallel with the Maxwell models, the $\mu_{m}$ are the
fraction of the shear modulus accommodated by each Maxwell model
spring, and $\epsilon^{\mathit{refdev}}_{ij}$ and
$\sigma^{\mathit{refdev}}_{ij}$ are the reference deviatioric strain
and stress, respectively. The viscous strain is:
\begin{equation}
h^{m}_{ij}\left(t\right)=\exp\frac{-\Delta
  t}{\tau_{m}}h^{m}_{ij}\left(t_{n}\right)+\Delta h^{m}_{ij},
\end{equation}
where $t_{n}$ is a time between $t=0$ and $t=t$, $\Delta
h^{m}_{ij}$ is the viscous strain between $t=t_{n}$ and
$t=t$, and $\tau_{m}$ is the Maxwell time:
\begin{equation}
  \tau_{m}=\frac{\eta_{m}}{\mu_{tot}\mu_{m}}.
\end{equation}
Approximating the strain rate as constant over each time step,
this is given as:
\begin{equation}
\Delta h^{m}_{ij}=\frac{\tau_{m}}{\Delta t}\left(1-\exp\frac{-\Delta
  t}{\tau_{m}}\right)\left(\epsilon^{\mathit{dev}}_{ij}\left(t\right)-\epsilon^{\mathit{dev}}_{ij}\left(t_{n}\right)\right)=\Delta
h^{m}\left(\epsilon^{\mathit{dev}}_{ij}\left(t\right)-\epsilon^{\mathit{dev}}_{ij}\left(t_n\right)\right).
\end{equation}
The approximation is singular for zero time steps, but a series
expansion may be used for small time-step sizes:
\begin{equation}
  \Delta h^{m}\approx1-\frac{1}{2}\left(\frac{\Delta
    t}{\tau_{m}}\right)+\frac{1}{3!}\left(\frac{\Delta
    t}{\tau_{m}}\right)^{2}-\frac{1}{4!}\left(\frac{\Delta
    t}{\tau_{m}}\right)^{3}+\cdots\,.
\end{equation}
This converges with only a few terms.
