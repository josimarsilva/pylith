\section{Labels and Identifiers for Materials, Boundary Conditions, and Faults}

For materials, the ``label'' is a string used only for error messages.
The ``id'' is an integer that corresponds to the material identifier
in LaGriT (itetclr) and CUBIT/Trelis (block id). The id also tags the
cells in the mesh for associating cells with a specific material model
and quadrature rule.

For boundary conditions, the ``label'' is a string used to associate
groups of vertices (psets in LaGriT and nodesets in CUBIT/Trelis) with
a boundary condition. Some mesh generators use strings (LaGriT) to
identify groups of nodes while others (CUBIT/Trelis) use strings and
integers. The default behavior in PyLith is to use strings to identify
groups for both LaGriT and CUBIT/Trelis meshes, but the behavior for
CUBIT/Trelis meshes can be changed to use the nodeset id (see Section
\vref{sec:MeshIOCubit}).

PyLith 1.0 had an ``id'' for boundary conditions, but we removed it
from subsequent releases because it was not used. For faults the
``label'' is used in the same manner as the ``label'' for boundary
conditions. That is, it associates a string with a group of vertices
(pset in LaGriT and nodeset in CUBIT/Trelis). The fault ``id'' is a
integer used to tag the cohesive cells in the mesh with a specific
fault and auxiliary data. Because we use the fault ``id'' to tag
cohesive cells in the mesh the same way we tag normal cells to
materials, it must be unique among the faults as well as the
materials.


% End of file
